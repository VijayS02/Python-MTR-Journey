\documentclass[fontsize=11pt]{article}
\usepackage{amsmath}
\usepackage[utf8]{inputenc}
\usepackage{subcaption}
\usepackage[margin=0.75in]{geometry}
\setlength{\parskip}{1em}
\usepackage{CJKutf8}
\usepackage{graphicx}
\graphicspath{ {./images/} }



\title{CSC111 Project: Speed information for the Hong Kong MTR}
\author{Vijay Sambamurthy}
\date{Friday, 16th March, 2021}

\begin{document}
\maketitle

\section*{Introduction}
Note: This is mostly similar to the original project proposal, changes have been highlighted through italicized text.

Hong Kong is a city known for its fast paced lifestyle and amazing public transport. One of the most rapid features of Hong Kong is its MTR (Mass Transit Railway) system. This system carries almost half of Hong Kong's entire population to their destinations each day (MTR, 2021). 

\begin{figure}[h]

\begin{subfigure}{0.5\textwidth}
\includegraphics[width=0.9\linewidth, height=5cm]{mtrsystemmap} 
\caption{HK MTR System Map}
\label{fig:subim1}
\end{subfigure}
\begin{subfigure}{0.5\textwidth}
\includegraphics[width=0.9\linewidth, height=5cm]{hkroutemap}
\caption{MTR System Map based on real positions}
\label{fig:subim2}
\end{subfigure}

\caption{HK MTR system images}
\label{fig:image2}
\end{figure}

The MTR system is comprised of 11 different lines and over 98 stations. During this research, the light rail line will be excluded as it is a completely different type of train, it is also much less commonly used and is generally not quite considered part of the MTR (it has its own system map). Each line operates at different speeds and frequencies. The MTR operates very similarly to the London underground in terms of structure. Also note that MTR is partially privatized corporation and runs franchised contracts to countries and cities all over the world including London, Shenzen, and Sydney.

Most people using the system rely on its efficiency and organization to get to their destination on time. However, train information and timings are not a concern for most of the population due to the 1-2 minute frequency of most lines on the Hong Kong system. But, knowing this information to an accurate extent (less than 5 minutes of error) could make people more dependant on the MTR and allow them to fit tighter schedules because they can easily account for traveling time. The Hong Kong MTR has delays so rarely that whenever such events occur, they can easily be expected to be the front page of every local news outlet. 


Personally, there was interest as to finding out the timings of the system and possibly building an app to give information on how long a journey would take.\textit{ Note that there is already an MTR app which provides this service, however, I am curious as to how accurate it is. Unfortunately, in the robots.txt of the MTR website, scraping data from the journey planner is not allowed, meaning that comparisons will have to be done manually.} Leading me to the question, \textit{\textbf{How long does a given Hong Kong MTR journey approximately take?}} This question is important in the context of Hong Kong (due to the rushed nature of the city) and transportation in general due to the reasons explained above. 

\textit{In order to investigate the Hong Kong MTR and find out possible route times, the speed of each line, latitude, and longitude data will be taken from Wikipedia's API in order to calculate distances between certain stations. }





\section*{Dataset Description}
The starting point for the program is the following dataset taken from \url{https://data.gov.hk/en-data/dataset/mtr-data-routes-fares-barrier-free-facilities}.
While there is no public location information available data for the MTR stations. Anyone can acces data containing a list of all MTR stations, route fares, and station IDs.  Some example rows of the "MTR Lines (except Light Rail) & Stations" dataset provided by MTR is: 
\begin{center}
\begin{tabular}{||c | c | c | c | c | c | c | c||} 
\hline
Line Code & Direction & Station Code & Station ID & Chinese Name & English Name & Sequence\\ [0.5ex] 
\hline \hline
AEL & DT & AWE & 56 & \begin{CJK*}{UTF8}{bsmi} 博覽館 \end{CJK*} & AsiaWorld-Expo & 1.00 \\
\hline
EAL & LMC-DT & FOT & 69 & \begin{CJK*}{UTF8}{bsmi} 火炭 \end{CJK*} & Fo Tan & 7.00 \\
\hline
KTL & DT & MOK & 6 & \begin{CJK*}{UTF8}{bsmi} 旺角 \end{CJK*} & Mong Kok & 14.00\\ [1ex] 
\hline
\end{tabular}
\end{center}
While most of this is not important, it can be combined with information from wikipedia which, contains coordinate information, to create a new dataset. This new dataset that will be created will contain: Line Code, Direction, Station Code, Station ID, Chinese Name, English Name, Sequence, Latitude, and Longitude. Each station will have multiple entries, 2 for each line it is on because of the "direction" variable that MTR has provided. This direction describes whether the direction is "up track" or "down track". 


In order to get the latitude and longitude data, wikipedia has an API (\url{https://en.wikipedia.org/w/api.php}). This can then be used to get a JSON response containing an object which will contain the coordinates of the given looked up station. All of this information will be stored in a CSV file called \textit{modified\_stations\_and\_lines.csv}


Each of the lines that will be used will have research conducted by me separately to find the operating speed of the train and the acceleration times. This data will be stored in a CSV file called \textit{lines.csv}. This contains: Line code, Operating speed (in Kmph), Line English Name for all lines on the network. It also contains an extra line "walking" which is used for the path between Hong Kong and Central. 

A second file created by me, labeled \textit{append.csv} contains information for the walkway between Central and Hong Kong station. This is required because there is a walkway between Central and Hong Kong station that is inside the paid area (area where the train travellers can interchange lines). By creating this append file, I am pretty much simulating a new line. This will be appended to the modified lines and stations csv data.

Finally, a third file created by me, titled \textit{filter.csv} is useed to fix any typos or changes in the station names. This was created because there is a typo in the station "Whampoa" which is simply listed as "Whampo" on the \textit{stations\_and\_lines.csv}

Python will then use the \textit{lines.csv} file in conjunction with the data from \textit{modified\_stations\_and\_lines.csv} to calculate "weights" which are actually the time taken to go between adjacent stations. This calculation will be done by adding 1 minute to the time taken to travel the distance between the two stations. This one minute is simply an approximation of the time it takes for the MTR to get up to speed and the time taken for it to decelerate to a stop. The distance traveled in this time is negligible. This approximation of 1 minute is taken from me manually timing a few lines during my journeys on them in the past week. ( See references )

All of this information will be stored in \textit{modified\_stations\_and\_lines\_APPENDED.csv} and \textit{lines.csv}. 

After the new dataset has been created, python can then take the newly created CSV file and consider each station as a node on the graph of the entire MTR station map. After this, operations can be easily done to calculate shortest paths.  



\section*{Computational Overview}
\textbf{Note: The reason for the separation between airport express and regular lines is that airport express is significantly more expensive and most people do not travel on the line unless they need to get to the airport.} 


Before going into the functions executed by the program itself, we must first look at \textit{classes.py} which contains 3 classes. The first of which is a \textbf{station} object, this object stores station data including english names, coordinates, neighbours, line codes and more. There are 3 main methods for this class, \textit{add\_line}, which simply adds a line code to the station's line codes. \textit{add\_neighbour}, which adds a neighbour to either airport express neighbours or regular neighbours depending on the parameters of the method. And finally, \textit{get\_weight} which returns the edge weight between this station and the station specified in the parameter. This function will only look at airport express neighbours if specified to in the parameters.

The next class is the \textbf{line} object, which contains simple line information like line code, english name for the line, stations on the line and operating speeds. Stations that are on the line are stored based on the sequence at which they are seen on the line. For example, the Tung Chung Line has the following stations mapping (each entry in the value's list is a station object):\\\\ \{1: [Tung Chung station], 2: [Sunny Bay station], 3: [Tsing Yi Station], 4: [Lai King station], 5: [Nam Cheong station] ...\}\\\\
This was done because of the case of the East rail line and the Tseung Kwan O line, which both have multiple stations at a given sequence on the line. 
\begin{figure}[h]

\begin{subfigure}{0.5\textwidth}
\includegraphics[width=0.9\linewidth, height=5cm]{east_rail_multiple} 
\caption{Multiple Starting stations on the East Rail Line}
\label{fig:subim1}
\end{subfigure}
\begin{subfigure}{0.5\textwidth}
\includegraphics[width=0.9\linewidth, height=5cm]{tko_multiple}
\caption{Multiple stations along the Tseung Kwan O line}
\label{fig:subim2}
\end{subfigure}

\caption{HK MTR system Ambiguities}
\label{fig:image2}
\end{figure}

The \textbf{line} class also has two methods, the first of which is the \textit{add\_station} method, which adds a station to the line. Next, is the \textit{add\_connection} method, which simply makes the 2 stations specified in the parameters have each other as neighbours (possibly airport express neighbours if the line is the Airport Express).

Finally, the \textbf{SystemMap} class (graph) is comprised of a dictionary of lines and a dictionary of stations and stores all the stations. The \textit{add\_station} method copies the properties of the station to an existing version if it already exists, otherwise it simply adds a new station. The \textit{add\_line} method simply adds the line to the lines mapping and adds every station in the line using the \textit{add\_station method}.

\\\\
Finally, the main method of the program, \textit{\textit{dijkstra}}. This algorithm is an implementation of dijkstra's shortest path algorithm, which uses a priority queue and generates the shortest path between 2 nodes on a graph. Here, it generates the shortest path between 2 stations on the system. A priority queue is used here because the running time of dijkstra's algorithm is heavily dependant on the \textit{decrease key} and \textit{extract min} functions of the data type that is being used. A Fibonacci heap would have been ideal to use in this situation but is a significant amount of work to implement manually based on a List in python. Hence, \textit{heapq}, a module in python which is easy to use for what is required when implementing dijkstra's, was used. Unfortunately, there is no \textit{decrease key} function. So instead, this version of the algorithm simply adds a station to the heapq multiple times. And if it has been visited, it means that there was already a shorter path to that station found previously, and hence any later entries found in the priority queue can be ignored. 

In addition, this function accounts for whether the user would like to use airport express or not by looking at the airport express neighbours of a station if allowed. 

The function returns the optimal path between 2 stations.



The functions of the program is split up into 4 separate parts: \textit{Data collection, information processing, mapping, and the main program}. 
\begin{enumerate}
    \item \textbf{Data Collection}\\
    In data collection, the \textit{data\_collection} python file is used. This file simply loads in the \textit{mtr\_lines\_and\_stations.csv} file and then applies some processing.\\\\
    First, it removes the extra lines (Empty rows of a csv file that simply contain commas E.g. ",,,,,,") that are found at the end of the file. These blank lines seem to serve no purpose and hence are removed.\\
    Then, duplicate stations are removed. I.e. if a row of the dataset has a line code and station code pair that has already been accounted for, that entry in the dataset is ignored. \\
    After this, the main function of data collection is run. \textit{get\_station\_coordinates} is a function which takes in the dataset which now has no duplicates and then for each line a station is on, it sends a get request to a Wikipedia API using the \textit{requests} module and then appends the respective data to each station's row in the dataset. For example, if the current row is describing Hong Kong station, it finds Hong Kong station's coordinates through the API and then appends that to that row of Hong Kong station. \\\\
    Finally, the newly modified data is then written out to a file.
    \item \textbf{Information Processing}\\
    The \textit{information\_processing} file mainly contains functions that are used to load the data into their respective objects and is used to generate a System map from the csv data provided. 
    It does this by creating each line separately first, and then running a connections function which adds the connections between neighbouring stations. Finally, after each line is completed, it adds the entire line to the System Map. 
    \item \textbf{Mapping}
    This python file is used to inform python where the respective stations on the image of the mtr map are. It does this by displaying a station name using \textit{pygame} and then expecting the user to left click on where they would like the click box for that respective station to be. As a result, the coordinates and click boxes of the app can be easily changed whenever needed. It then writes out this data to a file titled \textit{coord\_mappings.csv}. See \textit{mapping.py} for more detailed information.
    \item \textbf{Main Program}
    The main program that runs the interactive UI through \textit{pygame} which displays a clickable MTR map. Before generating the UI however, it first prompts the user (in the console) for the units used to describe the weights in the System Map, i.e. the units for the edges between two stations on the network. After this, the user can then left click to select a source station and right click to select a destination station. Once a valid selection has been made, price information is looked up from the \textit{mtr\_lines\_fares.csv} and a shortest path is made. The program then displays the shortest path in the form of a blue line on the image and text beneath containing price and weight information. 
    
    In addition, the user can select whether or not they would like to use the airport express via the Airport Express button in the bottom right hand side of the screen.
    
    NOTE: if \textit{run\_main} is run with the last parameter (\textit{show\_station\_boxes}) set to true, then the click boxes of each station will be displayed on the screen.
\end{enumerate}

\textit{pygame} was used when making interactive elements because the positions for which clicks are searched can be easily modified from a csv file, making it easy to change station positions if required. In addition, the raw nature of \textit{pygame} allows a path to be easily drawn.

\newpage
\section*{Instructions for Obtaining Datasets}
Please ensure that all imports in \textit{requirements.txt} have been installed. 

The zip file uploaded to markus titled \textit{data.zip} contains all the data required for this program.

This file should be unziped at the same level as \textit{main.py} using something like 7zip's "Extract Here" function. This will make a new folder tilted data appear at the same level as \textit{main.py}. Within this folder should immediately be 9 items. Suppose \textit{main.py} is at the root directory "/main.py". Then the directory of the mtrmap.png should be "/data/mtrmap.png".

Some of the data in the provided zip file is automatically generated by the program but has been provided in case the program takes too long. \textit{mtr\_lines\_and\_stations.csv} and \textit{mtr\_lines\_fares.csv} are csv files downloaded from the MTR website. \textit{modified\_mtr\_lines\_and\_stations.csv}, {coord\_mappings.csv} and \textit{modified\_mtr\_lines\_and\_stations\_APPENDED.csv} are automatically generated by the programs if executed in the order described by computational overview. (i.e. data\_collection.py \to mapping.py \to main.py). The rest have been manually made by me.

PLEASE NOTE: \textit{mapping.py} takes a few minutes to complete, especially if the user is not as familiar with the MTR system. (e.g. Kowloon station may be confused for Kowloon Tong station)

Mapping.py's boxes should be placed something like the following image:
\begin{figure}[h]
		\caption{Example Result of Mapping.py}
	    \includegraphics[scale=0.5]{mapping_ex}
	    \centering
\end{figure}

If any mistake is made during mapping, press the right mouse to undo.
\\\\
\textbf{Main.py}\\\\
When running main.py, please enter the units that are required in the console. In addition, the \textit{run\_main} function's call in the "\_\_main\_\_" block can be altered to show the click boxes of each station.  

Then a window should pop up which shows the MTR map and some other information.

\section*{Changes from Proposal}
As mentioned at the start, changes in the introduction were italicized but will be reiterated here. 
The program now distinguishes between routes that are using airport express and routes that are not using the airport express. Not all fares are shown for each route, only the fare for an adult with an octopus card (This can be changed, see main.py's constants and \textit{run\_main} for more details).

\section*{Discussion}
After using the program that was made to check the time of some journeys that I travel regularly, I can say that it is fairly accurate in terms of MTR travel speed. However, once comparing it to the MTR's official times, I see some significant disparities. But, my father, who times some of his daily journeys has found that the MTR app does overestimate the time taken during travel by a significant amount. It does this to account for the fact that people traveling on the MTR will not have instant interchanges. Unfortunately, my program does not account for this and assumes that switching between every line (except Hong Kong to Central) does not take any time and that the trains are already there waiting. This is because it is hard to get raw timing information between two MTR stations without manually timing them as MTR does not provide this information. The method of using latitude and longitude data to find the distance between them assumes that the rail follows the "as the crow flies" path and that the MTR always travels at operating speeds between every two stations. This is not the case in reality. The MTR does not always travel at it's max operating speed. 

Nonetheless, the travel times are fairly accurate given that they do not account for interchange and waiting.

A significant difficulty encountered when making the app, was the programmatic importing of the CSV MTR data into the system graph because of the 2 special cases of Lohas park and Lok Ma Chau. 

In the future, an update to this program would be to account for interchange times and possibly using the real time train information provided by the MTR to calculate how much waiting is needed to be done if the journey began as soon as possible. Moreover, the graphics displaying the path to be taken is somewhat flimsy and could have been updated by possibly using a sort of masking method to show and hide certain parts of a secondary image which is a highlighted version of the MTR map. This would result in a more natural look to the UI.

\section*{References}

Commons.wikimedia.org. 2021. File:Hong Kong Railway Route Map blank.svg. online Available at:\\ https://commons.wikimedia.org/wiki/File:Hong\_Kong\_Railway\_Route\_Map\_blank.svg [Accessed 14 March 2021].

Mtr.com.hk. 2021. MTR > Patronage Updates. online Available at:\\ https://www.mtr.com.hk/en/corporate/investor/patronage.php [Accessed 11 March 2021].

Mtr.com.hk. 2021. MTR > System Map. online Available at:\\ http://www.mtr.com.hk/en/customer/services/system\_map.html [Accessed 13 March 2021].

Data.gov.hk. 2021. | DATA.GOV.HK. online Available at: \\
https://data.gov.hk/en-datasets/search/MTR [Accessed 13 March 2021].

https://en.wikipedia.org/wiki/Dijkstra%27s_algorithm

https://www.pygame.org/docs/

https://geopy.readthedocs.io/en/stable/

Station acceleration times (Note these values varied heavily from station to station, this is simply an average): \\
Tung Chung Line - 23.5 seconds\\
Tsueng Kwan O Line - 27.2 seconds\\
East Rail Line - 31.1 seconds\\
Kwun Tong Line - 30.2 seconds\\
\end{document}

